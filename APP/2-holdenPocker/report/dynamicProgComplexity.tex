\documentclass[a4paper,10pt]{article}

% This line indicates the type of the document betwin {}: here it is a scientific article.
% Options betwin [] are not mandatory, but precise here:
% - a4paper: printing paper format
% - 10pt: size of the characters
\usepackage{qtree}
\usepackage{graphicx}
% This package allows to include images
\usepackage{titling}
% This package allows to have a subtitle 
\usepackage{listings}
% this package is used to write code samples.
\usepackage{pseudocode}
\usepackage[T1]{fontenc}
\usepackage[french]{babel}
\usepackage{array,multirow,makecell}
\usepackage{tikz}
\usepackage{epigraph}



\setcellgapes{1pt}
\makegapedcells
\newcolumntype{R}[1]{>{\raggedleft\arraybackslash }b{#1}}
\newcolumntype{L}[1]{>{\raggedright\arraybackslash }b{#1}}
\newcolumntype{C}[1]{>{\centering\arraybackslash }b{#1}}


\lstset{%
  basicstyle=\scriptsize\sffamily,%
  commentstyle=\footnotesize\ttfamily,%
  frameround=trBL,
  frame=single,
  breaklines=true,
  showstringspaces=false,
  numbers=left,
  numberstyle=\tiny,
  numbersep=10pt,
  keywordstyle=\bf
}
\newcommand{\subtitle}[1]{%
  \posttitle{%
    \par\end{center}
    \begin{center}\large#1\end{center}
    \vskip0.5em}%
}
%%%%%%%%%%%%%%%%%%%%%%%%%%%%%%%%%%%%%%%%%%%%%%%%%%%%
% Raport Headers:
%%%%%%%%%%%%%%%%%%%%%%%%%%%%%%%%%%%%%%%%%%%%%%%%%%%%
\title{Mathematics for computer science}
\subtitle{Homework 1}
\author{SID-LAKHDAR Riyane}
\date{15/10/2015}

\begin{document}
% Beginning serious stuff.


\maketitle
% Actually prints title / subtitle / authors and dat into the document

\begin{abstract}
  blablbalaba
  lmds
  
\end{abstract}


\section{Scheme}
    Let's consider the following deck of N.   Our algorithme will associate to each subsequence of the deck a (******* Put your structure *********).\newline
    The set of considered subsequences are guiven by the following tree:\newline
    \begin{tikzpicture}[sibling distance=20em, every node/.style = {shape=rectangle, rounded corners, draw, align=center, top color=white, bottom color=blue!20}]]
    \node {(0, N-1)}
    child
    {
	node {(0, N-2)}
	child[sibling distance=10em]
	{
	    node [sibling distance=5em] {(0, N-3)}
	    child[sibling distance=5em]
	    {
		node [sibling distance=5em] {(0, N-4)}
	    }
	    child[sibling distance=5em]
	    {
		node [sibling distance=5em] {(1, N-3)}
	    }
	}
	child[sibling distance=10em]
	{
	    node [sibling distance=5em]{(1, N-2)}
	    child[sibling distance=5em]
	    {
		node [sibling distance=5em]{(1, N-3)}
	    }
	    child[sibling distance=5em]
	    {
		node [sibling distance=5em]{(2, N-2)}
	    }
	}
    }
    child
    {
	node {(1, N-1)}
	child [sibling distance=5em]
	{
	    node[sibling distance=5em] {(1, N-2)}
	    child[sibling distance=5em]
	    {
		node[sibling distance=5em] {(1, N-3)}
	    }
	    child[sibling distance=5em]
	    {
		node[sibling distance=10em] {(2, N-2)}
	    }
	}
	child[sibling distance=10em]
	{
	    node[sibling distance=5em]{(2, N-1)}
	    child[sibling distance=5em]
	    {
		node[sibling distance=5em]{(2, N-2)}
	    }
	    child[sibling distance=5em]
	    {
		node[sibling distance=5em]{(3, N-1)}
	    }
	}
    };
    \end{tikzpicture}
\end{document}