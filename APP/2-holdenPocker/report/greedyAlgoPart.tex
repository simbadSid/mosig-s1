
\documentclass[a4paper,10pt]{article}
% This line indicates the type of the document betwin {}: here it is a scientific article.
% Options betwin [] are not mandatory, but precise here:
% - a4paper: printing paper format
% - 10pt: size of the characters

\usepackage{graphicx}
% This package allows to include images
\usepackage{titling}
% This package allows to have a subtitle 
\usepackage{listings}
% this package is used to write code samples.
\usepackage{pseudocode}
\usepackage[french]{babel}
\usepackage[T1]{fontenc}
\usepackage{array,multirow,makecell}
\setcellgapes{1pt}
\makegapedcells
\newcolumntype{R}[1]{>{\raggedleft\arraybackslash }b{#1}}
\newcolumntype{L}[1]{>{\raggedright\arraybackslash }b{#1}}
\newcolumntype{C}[1]{>{\centering\arraybackslash }b{#1}}


\lstset{%
  basicstyle=\scriptsize\sffamily,%
  commentstyle=\footnotesize\ttfamily,%
  frameround=trBL,
  frame=single,
  breaklines=true,
  showstringspaces=false,
  numbers=left,
  numberstyle=\tiny,
  numbersep=10pt,
  keywordstyle=\bf
}
\newcommand{\subtitle}[1]{%
  \posttitle{%
    \par\end{center}
    \begin{center}\large#1\end{center}
    \vskip0.5em}%
}
%%%%%%%%%%%%%%%%%%%%%%%%%%%%%%%%%%%%%%%%%%%%%%%%%%%%
% Raport Headers:
%%%%%%%%%%%%%%%%%%%%%%%%%%%%%%%%%%%%%%%%%%%%%%%%%%%%
\title{Mathematics for computer science}
\subtitle{Homework 1}
\author{SID-LAKHDAR Riyane}
\date{15/10/2015}

\begin{document}
% Beginning serious stuff.


\maketitle
% Actually prints title / subtitle / authors and dat into the document

\begin{abstract}
  blablbalaba
  lmds
  
\end{abstract}


\section{Greedy algorithme}
    To solve the problem, we have first tried to consider it using a greedy point of view:\newline
    Let's consider that solving the problem is equivalent to find, at each step of the game, the best card to pick, regarding only the actual state of the game.
    This greedy method does not consider the possible evolutions of the game after this step.\newline
    In the following greedy algorithm description and analyze, we will first assume that the sister can use any strategy.   In a second time, we will determine the complexity and the failure cases of the algorithme when the sister is choosing the biggest card at any step.
    \subsection{Algorithme descritpion}
    \begin{lstlisting}
    boolean winIsPossible(cards       : array of size N containing cards
                                        guiven by them values
                          sisterStart : boolean)
    {
	int min         = 0;   // Lowest index of the deck in the input array
	int max         = N-1; // Highest index of the deck
	int sisterScore = 0;
	int myScore     = 0;

	// The variables used with "&" may be modified by the function
	if (sisterStarts) playSister(cards, &min, &max, &sisterScore);
	for (i=0; i< N/2; i++)
	{
	    d0 = cards[min] - Math.max(cards[min+1], cards[max]);
	    d1 = cards[max] - Math.max(cards[min]  , cards[max-1]);
	    if (|d0| < |d1|) {myScore += cards[min]; min++;}
	    else             {myScore += cards[max]; max--;}
	    if (min != max) playSister(cards, &min, &max, &sisterScore);
	}
    }
    \end{lstlisting}
    The function \" playSister \" will play according to the choosen sister's strategy.  It will also update all the game parameters.\newline
    Assuming that it is our turn to pick a card, we can pick one card belonging to \{cards[0], cards[N-1]\}.
    Given one card of this set, we can compute the points we will earn, and the points the sister can earn.
    The aim of our algorithm is to chose, at each, step the card that will make the difference between our gain and the sister's gain as big as possible.\newline

    This algorithm has many obvious advantages.  First of all, it is very easy to implement and to understand.  No specific data structure or algorithm are required.
    Secondly, as we will prove in the next paragraph, this algorithm seems to have an interesting time complexity.
    However, as we do not consider the whole aspects of the problem, we can wonder if this algorithm will always give an optimal (or at least a working) solution?

    \subsection{Time and space complexity of this greedy resolution}
    The algorithm realises in a N/2 length loop, 3 instruction with an O(1) complexity.
    Our algorithm complexity can be written as:
    \begin{equation}C(N) = \frac{N}{2} * 3*O(1) = O(N) \end{equation}
    Thus, the time complexity of our algorithm is linear.\newline
    For all the game steps, our algorithm uses the same set of 4 integers (in addition to the N cards array).\newline
    Thus, the space complexity of our algorithm is O(N + 4) = O(N)\newline
    \newline
    You may have noticed that in our complexity analyze, we do not consider the complexity of the sister's decision algorithm.  
    This is done on purpose, as the problem we deal with is to find an algorithm which determines our decision.  
    However, the next algorithms we have designed have to compute the sister's decision to evaluate our decisions.   
    Thus, to make the comparison between all our algorithm more faire, we will include the complexity of the sister's algorithm in the computation of the time complexity of our greedy algorithm:\newline
    Assuming that at each step the sister pick up the biggest card, we can rewrite our time complexity as follows:
    \begin{equation}C(N) = \frac{N}{2} * (3*O(1) + O(1)) = O(N) \end{equation}
    
    
    
    \subsection{Failure cases}
    Despite its easy implementation and its interesting complexity, the greedy algorithm we designed is not an acceptable way to solve the given problem.\newline
    An easy way to get convinced of this fact is to consider the following counterexample:\newline
    \begin{tabular}{| L{2cm} || C{1cm} | C{1cm} | C{1cm} | C{1cm} | C{1cm} | C{1cm} |}
      \hline Index & 0 & 1 & 2 & 3 & 4 & 5 \\
      \hline Cards & 4 & 4 & 6 & 6 & 12 & 9 \\
      \hline
    \end{tabular}
    \newline
    \newline
    Running our algorithm on this input deck would make us loose the game by choosing the following cards:\newline
    \begin{tabular}{| L{2cm} || C{2cm} | C{2cm} | C{2cm} | C{2cm} | C{2cm} | C{2cm} |}
      \hline Index & 0 & 1 & 2 & 3 & 4 & 5 \\
      \hline Cards & 4 & 4 & 6 & 6 & 12 & 9 \\
      \hline Player ID & Sister 3  & You 3 & Sister 2 & You 2 & Sister 1 & You 1\\
      \hline
    \end{tabular}
    \newline
    \newline
    However, a solution which could make us win exists (respecting the same player's order).\newline
    \begin{tabular}{| L{2cm} || C{2cm} | C{2cm} | C{2cm} | C{2cm} | C{2cm} | C{2cm} |}
      \hline Index & 0 & 1 & 2 & 3 & 4 & 5 \\
      \hline Cards & 4 & 4 & 6 & 6 & 12 & 9 \\
      \hline Player ID & You 1  & Sister 3 & You 3 & Sister 2 & You 2 & Sister 1\\
      \hline
    \end{tabular}
    \newline
    \newline
    This counterexample shows the limits of our method: By considering only the current card we can choose at a given step, we may allow the sister to access to a card which is much butter at the next step.\newline
    Thus, at a given step of the game, we should be able to consider the implications of the current choice on all stages of the game until the end.
    And that is the purpose of the improvement made by the next algorithm.

\end{document}